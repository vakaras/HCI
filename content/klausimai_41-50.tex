\begin{question}{%
  Palyginkite tiesioginių ir netiesioginių nurodymo įrenginių privalumus
  ir trūkumus.
  }{exam:question:41}

  Nurodymo įrenginių palyginimas\cite[39]{skaidres-16}:
  \begin{itemize}
    \item tiesioginiai įrenginiai greiti, bet netikslūs;
    \item netiesioginiai nurodo tiksliai, tačiau jiems reikalingas
      lygus paviršius;
    \item grafinė planšetė patraukli, dirbant su ja ilgesnį laiką;
    \item pelė yra spartesnė už trimatę vairasvirtę;
    \item rodyklių klavišai yra spartesni už pelę ir sumažina raumenų
      skausmą.
  \end{itemize}

  Tiesioginio nurodymo įrankiai:
  \begin{itemize}
    \item šviesos pieštuko trūkumai\cite[30]{skaidres-16}:
      \begin{itemize}
        \item ranka uždengia ekrano dalį;
        \item ranka atitraukiama nuo klaviatūros;
        \item reikia pasiimti pieštuką;
      \end{itemize}
    \item jutiklinio ekrano \en{touch screen}
      trūkumai\cite[31]{skaidres-16}:
      \begin{itemize}
        \item ranka greitai pavargsta;
        \item uždengtas ekranas;
        \item ranka patraukiama nuo klaviatūros;
        \item netikslu;
        \item ekranas greitai susitepa;
      \end{itemize}
  \end{itemize}
  Jų privalumai\cite[32]{skaidres-16}:
  \begin{itemize}
    \item natūralus nurodymas skystųjų kristalų paviršiuje;
    \item FIXME: matomumas – neuždengta aplinka;
    \item gestų ir rašmenų atpažinimas.
  \end{itemize}

  Netiesioginio nurodymo įrankiai:
  \begin{itemize}
    \item pelės privalumai\cite[33]{skaidres-16}:
      \begin{itemize}
        \item ranka patogioje padėtyje;
        \item klavišai lengvai spaudžiami;
        \item net ilgi judesiai atliekami sparčiai;
        \item tikslus nurodymas;
      \end{itemize}
    \item rutulinis manipuliatorius\cite[34]{skaidres-16} \en{trackball}
      – paprastai realizuotas, kaip sukamas kamuolys nuo 1 iki 6
      colių skersmens, kuris valdo žymeklį;
    \item jutiklinis kilimėlis\cite[34]{skaidres-16} \en{touchpad}.
    \item grafinė planšetė \en{graphics tablet} – lietimui jautrus
      paviršius ekrano išorėje;
    \item vairasvirtė \en{joystick}.
  \end{itemize}
\end{question}

\begin{question}{%
  Kuris yra tiesioginio nurodymo įrenginys:
  \begin{itemize}
    \item jutiminis ekranas;
    \item grafinė planšetė;
    \item pelė;
    \item vairasvirtė
  \end{itemize}
  ?
  }{exam:question:42}
  Jutiminis ekranas.
\end{question}

\begin{question}{%
  Išvardinkite bent penkis novatoriškus sąveikos įrenginius ir
  paaiškinkite jų paskirtį.
  }{exam:question:44}
  \begin{enumerate}
    \item Valdymas pėdomis (pedalai) – tinka perjungti specialiuose
      taikymuose. \cite[43]{skaidres-16}.
    \item Žvilgsnio sekimas \en{eyetracking} panaudojant vaizdo kamerą.
      Problema: žvilgsniu galima netyčia nusiųsti nepageidaujamą komandą.
      Padeda turintiems judėjimo negalią. \cite[44]{skaidres-16}.
    \item Laisvo judesio įrenginiai \en{Multiple-degrees-of-freedom
      devices}. Su jais galima nurodyti poziciją ir orientaciją erdvėje.
      Jie tinka virtualioje realybėje bei negalintiems valdyti klavaitūros
      ir pelės.
    \item Duomenų pirštinės \en{DataGlove} – naudojamos medicinoje.
    \item Jutiminis atsakas \en{haptic feedback}: gerai atkreipia
      dėmesį, taikomas žaidimuose.
    \item Dviejų rankų įvedimas \en{Bimanual input} – FIXME:
      įvedimas naudojant iš karto abi rankas.
    \item Visur esanti kompiuterija \en{Ubiquitous computing} –
      dėvimi kompiuteriai, papildytos realybės akiniai.
    \item Lytimosios \en{tangible} sąsajos.
    \item Delniniai įrenginiai \en{Handheld devices}.
  \end{enumerate}
  Žr.: \cite[42]{skaidres-16}.
\end{question}

\begin{question}{%
  Kuris klaviatūros išdėstymas pritaikytas naujokui, o kuris vidutiniškai
  patyrusiam naudotojui?
  }{exam:question:45}
  <++>

  Naujokui: ABCDE.\cite[10]{skaidres-16}

  Patyrusiam: LEKP?

\end{question}

\begin{question}{%
  Sąsajų kūrimo priemonių lygiai.
  }{exam:question:46}

  FIXME: Su šitom skaidrėm ryškiai yra kažkas sumalta. Eclipse juk nėra 
  biblioteka?

  FIXME: Kuo skiriasi antras ir trečias lygiai? Pagal skaidres, tai
  trečias lygis yra „widget toolkit“, o antras – tai grafinės priemonės
  skirtos „sumėtyti“ norimus „widget“ į lango šabloną?

  \begin{enumerate}
    \item Langinės sistemos lygis remia vaizdo programavimą. Pavyzdžiai:
      „Windows Graphical User Interface“, „Quartz Compositor“,
      „X Window System“, „Wayland“.\cite[20]{skaidres-17}
    \item Grafinės naudotojo sąsajos priemonių lygis remia vaizdo
      kūrimą. Pavyzdžiai: „Borland Jbuilder“, „Microsoft Builder Studio“,
      „Qt“, „Gtk“.
    \item Specializuotų kalbų / karkasų lygis remia sąsajos koncepcijos
      kūrimą. Pavyzdžiai: „Macromedia Director“, „Tcl/Tk“,
      „Microsoft MFC“.
    \item Dalykinės srities lygis remia modelių kūrimą. Pavyzdžiui,
      „Microsoft Access“, „Sybase PowerDesigner“, „Supple“ (automatinis
      sąsajų generatorius).
  \end{enumerate}
  Žr.: \cite[17]{skaidres-17}
\end{question}

\begin{question}{%
  Sąsajų kūrimo priemonių pasirinkimo kriterijai.
  }{exam:question:47}
  \begin{enumerate}
    \item Priemonės kuriama sistemos dalis.
    \item Priemonės išmokimo laikas.
    \item Kūrimo laikas.
    \item Siūloma kūrimo metodika.
    \item Komunikacija su kitomis sistemomis.
    \item Išplečiamumas ir moduliškumas.
  \end{enumerate}
  Žr.: \cite[32]{skaidres-17}
\end{question}
