\begin{question}{%
  Kokių mokslo disciplinų sandūroje atsirado žmogaus ir kompiuterio
  sąveika? Trumpai argumentuokite kiekvienos disciplinos indėlį.
  }{exam:question:01}
  \begin{enumerate}
    \item \emph{Inžinerija} Spartėjanti aparatūra ir geresnių
      sąsajų kūrimo priemonės.
    \item \emph{Dizainas} Sąsajos išdėstymas.
    \item \emph{Kalbotyra} Komandų kalba.
    \item \emph{Ergonomika, žmogaus faktoriai} Įrangos projektavimas.
    \item \emph{Dailė} Estetinis patrauklumas.
    \item \emph{Filosofija} Darnos sąvoka.
    \item \emph{Fiziologija} Fiziniai gebėjimai.
    \item \emph{Antropologija} Naudotojo kūno formos.
    \item \emph{Sociologija} Grupinio darbo priemonės.
    \item \emph{Dirbtinis intelektas} Naudotojo modelis. Pagalbos teikimas.
    \item \emph{Psichologija} Naudotojo modelis. Naudotojo poreikių
      suvokimas.
  \end{enumerate}
  Žr.: \cite[8p.]{konspektas} ir \cite[11--12]{konspektas}.
\end{question}

\begin{question}{%
  Kuo užsiima žmogaus ir kompiuterio sąveika?
  }{exam:question:02}
  \begin{description}
    \item[Žmogaus ir kompiuterio sąveika] yra mokslo disciplina, užsiimanti
      interaktyvių programų sistemų, skirtų žmogui naudoti,
      \begin{itemize}
        \item projektavimu,
        \item įvertinimu,
        \item realizavimu,
        \item šiuos procesus supančių reiškinių tyrimu.
      \end{itemize}
  \end{description}
  Žr.: \cite[8p.]{konspektas} ir \cite[25]{skaidres-1}.
\end{question}

\begin{question}{%
  Kodėl žmogaus ir kompiuterio sąveikos reikšmė didėja?
  }{exam:question:03}
  Esminė priežastis – internetas ir jo plėtra. 
  \begin{itemize}
    \item \emph{Vieša pasauliniu mąstu:}
      Internetas suteikia galimybę pritraukti platesnį potencialių
      naudotojų ratą. Todėl nežinia, kas ir kada pamatys
      sistemą/svetainę, bei visa tai skatina labiau domėtis
      panaudojamumo inžinerijos metodų kaštais ir nauda.
    \item \emph{Nebesudėtinga:}
      Atsiradus patogesniems ir paprastesniems kūrimo įrankiams,
      nebereikia būti dideliu specialistu, kad sukurti svetainę ar Web
      taikymą. Todėl daugiau žmonių tuom užsiima ir, norint būti
      konkurencingu, reikia įdėti  dagiau pastangų, kad padaryti
      gerai.
    \item \emph{Kūrimo greitis:}
      Kuriant sistemą kuo greičiau, svarbu naudoti pasiteisinančius
      panaudojamumo sprendimus.
    \item \emph{Neveikia „Paleidžiam, po to patobulinsim“:}
      Prieš interneto atsiradimą, sistemos  būdavo leidžiamos ne
      visai stabilios versijos, nes buvo mažesnė konkurencija ir
      klientai suvokdavo, kaip sistema veikia. Atsiradus internetui
      ypač svarbus tapo  pirmas įspūdis: klientas vieną kartą
      nusivylęs – antrą kartą nebeapsilankys.
    \item \emph{Daugiau nepatyrusių naudotojų:}
      Internetu galima atlikti vis daugiau dalykų, todėl daugėja
      naudotojų-naujokų. ŽKS reikšmė šiame punkte pasireiškia,
      jog pigiau iš karto pritaikyti nepatyrusiems nei perdaryti jau
      funkcionuojančią sistemą.
  \end{itemize}
  Žr.: \cite[28--29]{skaidres-1} ir \cite[38--48]{skaidres-1}.
\end{question}

\begin{question}{%
  Žmogaus akis prisitaiko prie dienos ir nakties matymo sąlygų. Smegenys
  interpretuoja matomą vaizdą. Akies centre esantys receptoriai
  (nebūtina prisiminti pavadinimo) pasižymi skirtingu jautrumu spalvoms.
  Kokios sąsajos projektavimo rekomendacijos išplaukia iš šių faktų?
  Paaiškinkite 3 rekomendacijas.
  }{exam:question:04}
  FIXME: Iš paminėtų faktų išplaukia tik pirmoji rekomendacija, kurią
  galima suskaidyti į tris.
  \begin{itemize}
    \item
      Žmogaus regos centre mažiausias jautrumas yra mėlynai spalvai.
      Mėlyna – gera spalva fonui, tačiau vengtina teksto rašymui ar
      linijoms. Reiktų vengti gretimų spalvų, besiskiriančių
      melsvumo kiekiu, nes jų kraštai susilieja. Labiausiai akis yra
      jautri raudonos spalvos atspalviams, o mėlynos spalvos
      atspalviams – mažiausiai. Priešingos spalvos neturėtų būti
      kartu, nes tai vargina akis, ypač raudonos ir mėlynos bei
      geltonos ir violetinės spalvų deriniai.
    \item
      Dalis žmonių (iki 7\%) neskiria raudonos ir žalios, todėl derėtų
      vengti žalios ir raudonos spalvų atspalvių fonui ir pirmam planui.
      Laikui bėgant jautrumas kontrastui mažėja. Todėl projektuojant
      sistemas, skirtas vyresniems naudotojams, spalvų kontrastas turi
      būti didesnis nei jauniems žmonėms, būtent kompiuteriniuose
      žaidimuose.
    \item
      Per mažai spalvų yra geriau nei per daug. Vaikams skirtos
      programos turi būti spalvingesnės nei suaugusiems, nes spalvos
      skatina vaikų vaizduotę. Spalvų nauda (dėmesio atkreipimas,
      informacijos grupavimas) yra pametama, kai spalvų yra per daug.
      Skirtingų spalvų kiekis negali viršyti šešių.
  \end{itemize}
  Žr. \cite[28--29]{konspektas}
\end{question}

\begin{question}{%
  Žmogus turi tris atminties tipus. Jie skiriasi informacijos išlaikymo
  trukme ir saugojimo būdais. Pateikite dviejų tipų esmines
  charakteristikas.
  }{exam:question:05}
  \begin{enumerate}
    \item \emph{Trumpalaikė (darbinė):}
      išsilaiko trumpai (vidutiniškai 10 sek.); veikia mąstymo metu;
      greitai pasiekiama ir greitai irstanti; po kelių sekundžių
      perkeliama į ilgalaikę atmintį; pasiekiama porcijomis.
    \item \emph{Ilgalaikė:}
      didelė, beveik neribota; informacija saugoma semantinėse
      grandinėse; svarbus įsiminimo kontekstas; gali būti atkuriama
      po ilgo laiko; informacija atgaminama lėčiau; ilgai nenaudota
      informacija užmirštama.
    \item \emph{Juntamoji:}
      Apdoroja jutimo organų signalus. Pirminis juntamosios
      informacijos užrašymas atminties sistemoje. Mes turime labai
      trumpą fotografinę atmintį, kuri vadinama atvaizdžio
      atmintimi. Akimirkai akys registruoja tikslią reginio pateiktį
      ir mes galime stulbinančiai tiksliai prisiminti bet kurią to
      reginio dalį. Tačiau tai trunka tik kelias dešimtąsias
      sekundės dalis.
  \end{enumerate}
  Žr.: \cite[33--34]{konspektas} ir \cite[21--25]{skaidres-3}.
\end{question}

\begin{question}{%
  Panaudojamumo apibrėžtis pagal ISO 9241.
  }{exam:question:06}
  \begin{description}
    \item[Panaudojamumas] — tai naudotojo veiklos
      \textbf{veiksmingumas} (efektyvumas), \textbf{našumas} ir
      \textbf{jaučiamas malonumas}, su kuriuo \textbf{konkretus}
      naudotojas gali pasiekti \textbf{konkrečių} tikslų
      \textbf{konkrečiose} aplinkose.
    \item[Efektyvumas] – tai naudotojo pasiekiamų tikslų
      užbaigtumas ir tikslumas.
    \item[Pasitenkinimas] – tai sistemos naudojimo patogumas ir
      priimtinumas.
  \end{description}
  Žr.: \cite[46p.]{konspektas} ir \cite[7]{skaidres-4}.
\end{question}

\begin{question}{%
  Išvardinkite standartinius panaudojamumo tikslus gaminio gyvavimo ciklo
  etapams.
  }{exam:question:07}
  FIXME: Normaliai suformuluoti.
  \begin{itemize}
    \item \emph{Poreikių įvertinimas:} esminių sistemos savybių
      demonstravimas.
    \item \emph{Pirkimas:} patogus pasirinkimas ir greitas pirkimas.
    \item \emph{Diegimas:} kuo trumpesnis ir sėkmingesnis.
    \item \emph{Apmokymas:} nesudėtingas, mokymo medžiaga lengvai prieinama
      ir atitinkanti programą.
    \item \emph{Ribotas naudojimas:} naudotojai yra paliekami savarankiškam
      darbui, yra tikimasi, kad jie savo užduotis atliks lengviau,
      patogiau ir greičiau.
    \item \emph{Pilnas naudojimas:} išnaudojamos visos sistemos galimybės,
      yra svarbu, kad būtų nesunku pašinti iki tol nenaudotas funkcines
      galimybes.
    \item \emph{Pasiekiamos gaminio ribos:} kai naudotojai jau pilnai
      išnaudoja sistemą ir jų darbo pobūdis pradeda keistis, jie
      pradeda jausti esamos sistemos ribas.
    \item \emph{Įvertinimas:} remiantis, sukaupta patirtimi iš naujo
      įvertinami poreikiai.
  \end{itemize}
  Žr.: \cite[47--49]{konspektas} ir \cite[6]{skaidres-4}.
\end{question}

\begin{question}{%
  Panaudojamumo projektavimo principų taksonomija (Dix ir kiti).
  Principų tinkamo projektavimo ir pažeidimo pavyzdžiai. (Pakanka
  tekstinių aprašų.)
  }{exam:question:08}
  Panaudojamumo principai pagal Dix.
  \begin{enumerate}
    \item Išmokstamumas \en{learnability} – kaip lengvai naujokai gali
      išmokti efektyviai naudotis sistema.
      \begin{itemize}
        \item Nuspėjamumas \en{predictability}:
          \begin{itemize}
            \item veiksmo rezultatų numatymas;
            \item operacijų matomumas.
          \end{itemize}
          Pavyzdžiui, tą patį veiksmą aliekantį mygtuką galima
          pavadinti „Gerai“ ir „Siųsti duomenis“.
        \item Sintezavimas \en{synthesizability}:
          \begin{itemize}
            \item gebėjimas įvertinti buvusių veiksmų poveikį esamai
              būsenai;
            \item kiekvieno veiksmo efekto matomumas;
            \item nuoseklus atsakas.
          \end{itemize}
          Pavyzdžiui, grafinėje sąsajoje pertempus failą iš vieno katalogo
          į kitą, mes iš karto tai matome. (Skirtingai nuo komandinės
          eilutės, kur papildomai dar reikia peržiūrėti katalogo turinį.)
          Sintezavimas nuo nuspėjamumo skiriasi tuo, kad nuspėjamumas
          akcentuoja asmens gebėjimą nuspėti veiksmo rezultatą, kai
          sintezavimas yra naudotojo gebėjimas įvertinti buvusių veiksmų
          poveikį esamai sistemos būsenai.
        \item Atpažįstamumas \en{familiarity}: santykis tarp
          \begin{itemize}
            \item naudotojo esamų žinių ir
            \item žinių, reikalingų efektyviam sistemos naudojimui.
          \end{itemize}
        \item Apibendrinimas \en{generalizability}: pavyzdžiui,
          daugumoje programų kartojasi meniu „Failas“ ir „Keisti“.
        \item Darna \en{consistency}:
          \begin{itemize}
            \item vienoje sistemoje – panašūs veiksmai įvairiose sistemos
              vietose atliekami vienodai;
            \item platformoje;
            \item darbo aplinkoje – atitikimas platformos (operacinės
              sistemos) stiliui, žinios gautos naudojant kitas tos
              OS programas turi būti pritaikomos ir naujajai;
            \item metaforose – veiksmai su virtualiais objektais
              neturi prieštarauti realiame pasaulyje vykdomiems veiksmams.
          \end{itemize}
      \end{itemize}
    \item Lankstumas \en{flexibility} – naudotojo ir sistemos apsikeitimo
      informacija būdų įvairovė. (Užduoties atlikimo būdų įvairovė.)
      \begin{itemize}
        \item Dialogo iniciatyva \en{dialogue initiative} – produktas
          nevaržo naudotojo pasirenkant įvestį. (Orientuota į
          tai, kad dialogą inicijuoja naudotojas, o ne sistema.)
        \item Daugiagijiškumas \en{Multithreading} – sistemos gebėjimas
          remti naudotojo sąveiką su keliomis lygiagrečiai vykdomomis
          užduotimis. Gali būti:
          \begin{description}
            \item[susikertantis \en{concurrent}] – viena įvestis tinka
              kelioms užduotims;
            \item[lygiagretus \en{interleaving}] – viena įvestis konkrečiai
              užduočiai.
          \end{description}
        \item Užduočių perkėliamumas \en{task migrability} – galimybė
          perduoti užduoties valdymą sistemai. Pavyzdžiui, gramatikos
          tikrinimas ir autopilotas.
        \item Pakeičiamumas \en{substitutivity}:
          \begin{itemize}
            \item matai (coliai, centimetrai, metrai ir t.t.);
            \item įvesties ir išvesties alternatyvos;
            \item leidimas keisti užduoties formulavimo ar rodymo būdą.
          \end{itemize}
        \item Pritaikomumas \en{customizability} – individualizavimas.
      \end{itemize}
    \item Robastiškumas \en{robustness} – naudotojo rėmimo laipsnis,
      nustatant sėkmingą sistemos naudojimą ir tikslų įvertinimą.
      \begin{itemize}
        \item Matomumas \en{observability} – gebėjimas iš vaizdo įvertinti
          sistemos būseną.
        \item Atstatomumas \en{recoverability} – galimybė ištaisyti
          klaidą.
        \item Sistemos atsakas \en{responsivness}.
        \item Užduočių atitikimas realioms veikloms \en{task conformance}:
          \begin{itemize}
            \item pavadinimai ir veiksmų eiliškumas;
            \item matavimo kriterijai.
          \end{itemize}
      \end{itemize}
  \end{enumerate}
  Žr.: \cite[59--64]{konspektas} ir \cite[11--40]{skaidres-4}.
\end{question}
