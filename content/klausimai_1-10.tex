
\begin{question}{%
  Kokių mokslo disciplinų sandūroje atsirado žmogaus ir kompiuterio
  sąveika? Trumpai argumentuokite kiekvienos disciplinos indėlį.
  }{exam:question:01}
  <++>
\end{question}

\begin{question}{%
  Kuo užsiima žmogaus ir kompiuterio sąveika?
  }{exam:question:02}
  <++>
\end{question}

\begin{question}{%
  Kodėl žmogaus ir kompiuterio sąveikos reikšmė didėja?
  }{exam:question:03}
  <++>
\end{question}

\begin{question}{%
  Žmogaus akis prisitaiko prie dienos ir nakties matymo sąlygų. Smegenys
  interpretuoja matomą vaizdą. Akies centre esantys receptoriai
  (nebūtina prisiminti pavadinimo) pasižymi skirtingu jautrumu spalvoms.
  Kokios sąsajos projektavimo rekomendacijos išplaukia iš šių faktų?
  Paaiškinkite 3 rekomendacijas.
  }{exam:question:04}
  <++>
\end{question}

\begin{question}{%
  Žmogus turi tris atminties tipus. Jie skiriasi informacijos išlaikymo
  trukme ir saugojimo būdais. Pateikite dviejų tipų esmines
  charakteristikas.
  }{exam:question:05}
  <++>
\end{question}

\begin{question}{%
  Panaudojamumo apibrėžtis pagal ISO 9241.
  }{exam:question:06}
  <++>
\end{question}

\begin{question}{%
  Išvardinkite standartinius panaudojamumo tikslus gaminio gyvavimo ciklo
  etapams.
  }{exam:question:07}
  <++>
\end{question}

\begin{question}{%
  Panaudojamumo projektavimo principų taksonomija (Dix ir kiti).
  Principų tinkamo projektavimo ir pažeidimo pavyzdžiai. (Pakanka
  tekstinių aprašų.)
  }{exam:question:08}
  <++>
\end{question}

\begin{question}{%
  Sėkmingo produkto savybės.
  }{exam:question:09}
  <++>
\end{question}

\begin{question}{%
  Tikslinio projektavimo etapai.
  }{exam:question:10}
  <++>
\end{question}
