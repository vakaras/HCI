
\begin{question}{%
  Kokių mokslo disciplinų sandūroje atsirado žmogaus ir kompiuterio
  sąveika? Trumpai argumentuokite kiekvienos disciplinos indėlį.
  }{exam:question:01}
  \begin{enumerate}
    \item \emph{Inžinerija} Spartėjanti aparatūra ir geresnių interfeisų kūrimo priemonės.
    \item \emph{Dizainas} Interfeiso išdėstymas.
    \item \emph{Kalbotyra} Komandų kalba.
    \item \emph{Ergonomika, žmogaus faktoriai} Įrangos projektavimas.
    \item \emph{Dailė} Estetinis patrauklumas.
    \item \emph{Filosofija} Darnos sąvoka.
    \item \emph{Fiziologija} Fiziniai gebėjimai.
    \item \emph{Antropologija} Naudotojo kūno formos.
    \item \emph{Sociologija} Grupinio darbo priemonės: .
    \item \emph{Dirbtinis intelektas} Naudotojo modelis. Pagalbos teikimas.
    \item \emph{Psichologija} Naudotojo modelis. Naudotojo poreikių suvokimas.
  \end{enumerate}
  <++>
\end{question}

\begin{question}{%
  Kuo užsiima žmogaus ir kompiuterio sąveika?
  }{exam:question:02}
  \begin{description}
    \item[Žmogaus ir kompiuterio sąveika] yra mokslo disciplina, užsiimanti
      interaktyvių programų sistemų, skirtų žmogui naudoti,
      \begin{itemize}
        \item{projektavimu}
        \item{įvertinimu}
        \item{realizavimu}
        \item{šiuos procesus supančių reiškinių tyrimu}
      \end{itemize}
  \end{description}
  <++>
\end{question}

\begin{question}{%
  Kodėl žmogaus ir kompiuterio sąveikos reikšmė didėja?
  }{exam:question:03}
  Esminė priežastis - internetas ir jo plėtra. 
  \begin{itemize}
    \item{Vieša pasauliniu mąstu: internetas suteikia galimybę 
      pritraukti platesnį potencialių vartotojų ratą.
      Todėl nežinia, kas ir kada pamatys sistemą/svetainę, bei visa tai skatina labiau
      domėtis panaudojamumo inžinerijos metodų kaštais ir nauda.}
    \item{Nebesudėtinga: atsiradus patogesniems ir paprastesniems kūrimo įrankiams, 
      nebereikia būti dideliu specialistu, kad sukurti svetainę ar Web taikymą.
      Todėl daugiau žmonių tuom užsiima ir, norint būti konkurencingu, reikia įdėti 
      dagiau pastangų, kad padaryti gerai.
      }
    \item{Kūrimo greitis: kuriant sistemą kuo greičiau, svarbu naudoti pasiteisinančius
      panaudojamumo sprendimus.
      }
    \item{Neveikia „Paleidžiam, po to patobulinsim“: prieš interneto atsiradimą, sistemos 
      būdavo leidžiamos ne visai stabilios versijos, nes buvo mažesnė konkurencija ir 
      klientai suvokdavo, kaip sistema veikia. Atsiradus internetui ypač svarbus tapo 
      pirmas įspūdis: klientas vieną kartą nusivylęs - antrą kartą nebeapsilankys.
      }
    \item{Daugiau nepatyrusių naudotojų: internetu galima atlikti vis daugiau dalykų,
      todėl daugėja vartotojų-naujokų. ŽKS reikšmė šiame punkte pasireiškia, jog pigiau
      iš karto pritaikyti nepatyrusiems nei perdaryti jau funkcionuojančią sistemą.
      }
  \end{itemize}
  <++>
\end{question}

\begin{question}{%
  Žmogaus akis prisitaiko prie dienos ir nakties matymo sąlygų. Smegenys
  interpretuoja matomą vaizdą. Akies centre esantys receptoriai
  (nebūtina prisiminti pavadinimo) pasižymi skirtingu jautrumu spalvoms.
  Kokios sąsajos projektavimo rekomendacijos išplaukia iš šių faktų?
  Paaiškinkite 3 rekomendacijas.
  }{exam:question:04}
  FIXME: ar tikrai tinka būtent šitos trys rekomendacijos??
    \begin{itemize}
      \item{
        Žmogaus regos centre mažiausias jautrumas yra mėlynai spalvai. Mėlyna - gera
        spalva fonui, tačiau vengtina teksto rašymui ar linijoms.
        Reiktų vengti gretimų spalvų, besiskiriančių melsvumo kiekiu, nes jų kraštai 
        susilieja. Labiausiai akis yra jautri raudonos spalvos atspalviams, o mėlynos 
        spalvos atspalviams – mažiausiai. Priešingos spalvos neturėtų būti kartu, nes 
        tai vargintų akis, ypač raudonos ir mėlynos bei geltonos ir violetinės spalvų 
        derinius.
        }
      \item{
        Neretai yra neskiriamos raudona ir žalia spalvos, todėl vengiame žalio ir 
        raudonos spalvų atspalvių fonui ir pirmam planui.
        Laikui bėgant jautrumas kontrastui mažėja. Todėl projektuojant sistemas, 
        skirtas vyresniems naudotojams, spalvų kontrastas turi būti didesnis nei jauniems 
        žmonėms, būtent kompiuteriniuose žaidimuose.
        }
      \item{Per mažai spalvų yra geriau nei per daug. Vaikams  skirtos programos 
        turi būti spalvingesnės nei suaugusiems, nes spalvos skatina vaikų vaizduotę. 
        Spalvų nauda (dėmesio atkreipimas, informacijos grupavimas) yra pametama, kai 
        spalvų yra per daug. Skirtingų spalvų kiekis negali viršyti šešių.
        }
    \end{itemize}
  <++>
\end{question}

\begin{question}{%
  Žmogus turi tris atminties tipus. Jie skiriasi informacijos išlaikymo
  trukme ir saugojimo būdais. Pateikite dviejų tipų esmines
  charakteristikas.
  }{exam:question:05}
  \begin{enumerate}
    \item \emph{Trumpalaikė (darbinė)} Išsilaiko trumpai (vidutiniškai 10 sek.); 
      veikia mąstymo metu; greitai pasiekiama ir greitai irstanti; po kelių sekundžių 
      perkeliama į ilgalaikę atmintį; pasiekiama porcijomis.
    \item \emph{Ilgalaikė} Didelė, beveik neribota; informacija saugoma semantinėse
      grandinėse; svarbus įsiminimo kontekstas; gali būti atkuriama po ilgo laiko;
      informacija atgaminama lėčiau; ilgai nenaudota informacija užmirštama.
    \item \emph{Juntamoji} Apdoroja jutimo organų signalus. Pirminis juntamosios 
      informacijos užrašymas atminties sistemoje. Mes turime labai trumpą fotografinę 
      atmintį, kuri vadinama atvaizdžio atmintimi. Akimirkai akys registruoja tikslią
      reginio pateiktį ir mes galime stulbinančiai tiksliai prisiminti bet kurią to 
      reginio dalį. Tačiau tai trunka tik kelias dešimtąsias sekundės dalis.
  \end{enumerate}
  <++>
\end{question}

\begin{question}{%
  Panaudojamumo apibrėžtis pagal ISO 9241.
  }{exam:question:06}
  \begin{description}
    \item[Panaudojamumas] — tai naudotojo veiklos \textbf{veiksmingumas} (efektyvumas),
      \textbf{našumas} ir \textbf{jaučiamas malonumas}, su kuriuo \textbf{konkretus} 
      naudotojas gali pasiekti \textbf{konkrečių} tikslų \textbf{konkrečiose} aplinkose.
    \item[Efektyvumas] – tai naudotojo pasiekiamų tikslų užbaigtumas ir tikslumas.
    \item[Pasitenkinimas] – tai sistemos naudojimo patogumas ir priimtinumas.
  \end{description}
  <++>
\end{question}

\begin{question}{%
  Išvardinkite standartinius panaudojamumo tikslus gaminio gyvavimo ciklo
  etapams.
  }{exam:question:07}
  \begin{enumerate}
    \item \emph{Efektyvumas} nurodo, kaip gerai sistema  atlieka tai, kam ji skirta. 
      Šis tikslas nusako, ar sistema suteikia naudotojams galimybes išmokti ją 
      naudoti ir pasiekti reikiamų rezultatų.
    \item \emph{Našumas} nurodo naudotojo veiklų rėmimo būdą. Teigiamas našumo pavyzdys: 
      elektroninės komercijos tinklapiuose vieną kartą įvesti asmininiai duomenys 
      perkant vieną prekę, kitame pirkime toje pačioje įmonėje duomenys bus įstatomi 
      automatiškai.
    \item \emph{Saugumas} reiškia, kad naudotojas yra apsaugotas  rizikingose ir 
      nepageidaujamose situacijose. Pavyzdžiui, rentgeno aparato operatorius vykdo 
      operacijas už sienos.
    \item \emph{Naudingumas}  Ar sistema suteikia visas reikalingas priemones 
      atlikti užduotį? Pavyzdys – galingos buhalterinės sistemos. Mažos naudos 
      pavyzdys – piešimo priemonė, kurioje nėra galimybės nupiešti laisvą liniją.
    \item \emph{Išmokstamumas}
    \item \emph{Lengvas atsimenamumas}
  \end{enumerate}
  <++>
\end{question}

\begin{question}{%
  Panaudojamumo projektavimo principų taksonomija (Dix ir kiti).
  Principų tinkamo projektavimo ir pažeidimo pavyzdžiai. (Pakanka
  tekstinių aprašų.)
  }{exam:question:08}
  Panaudojamumo principai:
  \begin{enumerate}
    \item Išmokstamumas (learnability)
      \begin{itemize}
        \item Nuspėjamumas (predictability)
        \item Sintezavimas (synthesizability)
        \item Atpažįstamumas (familiarity)
        \item Apibendrinimas (generalizability)
        \item Darna (consistency)
          \begin{itemize}
            \item Vienoje sistemoje
            \item Platformoje
            \item Darbo aplinkoje
            \item Metaforose
          \end{itemize}
      \end{itemize}
    \item Lankstumas (flexibility)
      \begin{itemize}
        \item Dialogo iniciatyva (dialogue initiative)
        \item Daugiagijiškumas (Multithreading)
        \item Užduočių perkėliamumas (task migrability)
        \item Pakeičiamumas (substitutivity)
        \item Pritaikomumas (customizability)
      \end{itemize}
    \item Robastiškumas
      \begin{itemize}
        \item Matomumas (observability)
        \item Atstatomumas (recoverability)
        \item Sistemos atsakas (responsivness)
        \item Užduočių atlikimas (task conformance)
      \end{itemize}
  \end{enumerate}
  
  Aprašymai pagal fantaziją. Pavyzdžiui, netinkamas pateikimas - durys, kai neaišku stumti ar traukti, kur stumti... Geras pavyzdys - mygtukas, aiškus interaktyvus objektas.
  
  <++>
\end{question}

\begin{question}{%
  Sėkmingo produkto savybės.
  }{exam:question:09}
  <++>
\end{question}

\begin{question}{%
  Tikslinio projektavimo etapai.
  }{exam:question:10}
  <++>
\end{question}
