\chapter{Egzamino klausimai}

\newcounter{questioncounter}
\setcounter{questioncounter}{0}
\newenvironment{question}[2]
{
\refstepcounter{questioncounter}
\thequestioncounter{}. \hspace{3em}\emph{#1}\\
\label{#2}
}
{}

\begin{question}{%
  Kokių mokslo disciplinų sandūroje atsirado žmogaus ir kompiuterio
  sąveika? Trumpai argumentuokite kiekvienos disciplinos indėlį.
  }{exam:question:01}
  Žr.: \cite[8p.]{konspektas}.
  Taip pat \cite[11--12]{konspektas}.
\end{question}

\begin{question}{%
  Kuo užsiima žmogaus ir kompiuterio sąveika?
  }{exam:question:02}
  \cite[8p.]{konspektas}, apibrėžimas puslapio apačioje.
  Taip pat: \cite[25p.]{skaidres-1}.
\end{question}

\begin{question}{%
  Kodėl žmogaus ir kompiuterio sąveikos reikšmė didėja?
  }{exam:question:03}
  \cite[28--29]{skaidres-1}, \cite[38--48]{skaidres-1}.
\end{question}

\begin{question}{%
  Žmogaus akis prisitaiko prie dienos ir nakties matymo sąlygų. Smegenys
  interpretuoja matomą vaizdą. Akies centre esantys receptoriai
  (nebūtina prisiminti pavadinimo) pasižymi skirtingu jautrumu spalvoms.
  Kokios sąsajos projektavimo rekomendacijos išplaukia iš šių faktų?
  Paaiškinkite 3 rekomendacijas.
  }{exam:question:04}
  Žr. \cite[28--29]{konspektas}
\end{question}

\begin{question}{%
  Žmogus turi tris atminties tipus. Jie skiriasi informacijos išlaikymo
  trukme ir saugojimo būdais. Pateikite dviejų tipų esmines
  charakteristikas.
  }{exam:question:05}
  \cite[33--34]{konspektas}
  \cite[21--25]{skaidres-3}
\end{question}

\begin{question}{%
  Panaudojamumo apibrėžtis pagal ISO 9241.
  }{exam:question:06}
  <++>
\end{question}

\begin{question}{%
  Išvardinkite standartinius panaudojamumo tikslus gaminio gyvavimo ciklo
  etapams.
  }{exam:question:07}
  <++>
\end{question}

\begin{question}{%
  Panaudojamumo projektavimo principų taksonomija (Dix ir kiti).
  Principų tinkamo projektavimo ir pažeidimo pavyzdžiai. (Pakanka
  tekstinių aprašų.)
  }{exam:question:08}
  <++>
\end{question}

\begin{question}{%
  Sėkmingo produkto savybės.
  }{exam:question:09}
  <++>
\end{question}

\begin{question}{%
  Tikslinio projektavimo etapai.
  }{exam:question:10}
  <++>
\end{question}

\begin{question}{%
  Naudotojų tipų poreikiai.
  }{exam:question:11}
  <++>
\end{question}

\begin{question}{%
  Personų paskirtis tiksliniame projektavime. Personos aprašo struktūra.
  }{exam:question:12}
  <++>
\end{question}

\begin{question}{%
  Palyginkite naudotojo potyrių ir panaudojamumo tikslus.
  }{exam:question:13}
  <++>
\end{question}

\begin{question}{%
  Naudotojui palankus projektavimas.
  }{exam:question:14}
  <++>
\end{question}

\begin{question}{%
  Hierarchinė užduočių analizė.
  }{exam:question:15}
  <++>
\end{question}

\begin{question}{%
  Maketavimo paskirtis.
  }{exam:question:16}
  <++>
\end{question}

\begin{question}{%
  Paaiškinkite Gareto naudotojo potyrių projektavimo sluoksnius.
  }{exam:question:17}
  <++>
\end{question}

\begin{question}{%
  Kaip naudotojui palankus projektavimas vykdomas stambiuose projektuose?
  Pateikite atitinkamų metodikų pavyzdžius.
  }{exam:question:18}
  <++>
\end{question}

\begin{question}{%
  Paaiškinkite Normano sąveikos etapų modelį ir indėlį sąsajos
  projektavimui.
  }{exam:question:19}
  <++>
\end{question}

\begin{question}{%
  Keturi gero projektavimo principai išplaukiantys ir Normano sąveikos
  etapo modelio.
  }{exam:question:20}
  <++>
\end{question}

\begin{question}{%
  Tarkime, turite sąsajos maketą ir galite pasirinkti \emph{tik vieną}
  iš žemiau pateiktų metodų:
  \begin{itemize}
    \item euristinis tikrinimas;
    \item pažintinė peržvalga;
    \item testavimas su naudotojais.
  \end{itemize}
  Argumentuokite savo pasirinkimą.
  }{exam:question:21}
  <++>
\end{question}

\begin{question}{%
  Tarkime, turite interaktyvų prototipą ir galite pasirinkti \emph{tik
  vieną} ir žemiau pateiktų metodų:
  \begin{itemize}
    \item euristinis tikrinimas;
    \item pažintinė peržvalga;
    \item testavimas su naudotojais.
  \end{itemize}
  Argumentuokite savo pasirinkimą.
  }{exam:question:22}
  <++>
\end{question}

\begin{question}{%
  Kuriuose vertinimo metoduose vertintojas bando numatyti, ar naujokas
  lengvai išmoks naudotis sistema:
  \begin{itemize}
    \item euristiniame tikrinime;
    \item pažintinėje peržvalgoje;
    \item KLM metode;
    \item testavime?
  \end{itemize}
  (Pasirinkite visus tinkančius atsakymus.)
  }{exam:question:23}
  <++>
\end{question}

\begin{question}{%
  Paaiškinkite GOMS ir KLM santykį.
  }{exam:question:24}
  <++>
\end{question}

\begin{question}{%
  Kas yra ir kokio naudotojo tipo veikimą tikrina veiksmų analizės metodai?
  }{exam:question:25}
  <++>
\end{question}

\begin{question}{%
  Palyginkite panaudojamumo vertinimus: modelinius ir ekspertų apžvalgas,
  konkrečius metodus, pavyzdžiui, pažintinę peržvalgą, euristinį
  tikrinimą, testavimą su naudotojais šiais aspektais:
  \begin{itemize}
    \item paskirtis,
    \item kokiuose projekto etapuose tikslinga vykdyti,
    \item vertinimo dalyviai,
    \item ką reikia paruošti vertinimui,
    \item kas yra vertinimo rezultatas
  \end{itemize}
  ?
  }{exam:question:26}
  <++>
\end{question}

\begin{question}{%
  Ką teigia Fito dėsnis? (Paaiškinkite esmę, formulės pateikti nereikia.)
  }{exam:question:27}
  <++>
\end{question}

\begin{question}{%
  Ką teigia Hiko dėsnis? (Paaiškinkite esmę, formulės pateikti nereikia.)
  }{exam:question:28}
  <++>
\end{question}

\begin{question}{%
  Palyginkite darbalaukinių ir kitokio tipo sistemų euristikas.
  }{exam:question:29}
  <++>
\end{question}

\begin{question}{%
  Kokios euristikos aktualios:
  \begin{itemize}
    \item naujokui,
    \item vidutiniškai patyrusiam,
    \item patyrusiam
  \end{itemize}
  naudotojui?
  }{exam:question:30}
  <++>
\end{question}

\begin{question}{%
  Kaip nustatomi testavimo metu aptiktų klaidų prioritetai?
  }{exam:question:31}
  <++>
\end{question}

\begin{question}{%
  Paaiškinkite netiesioginius vertinimo su naudotojais būdus.
  }{exam:question:32}
  <++>
\end{question}

\begin{question}{%
  Interneto svetainių euristikos.
  }{exam:question:33}
  <++>
\end{question}

\begin{question}{%
  Kas yra informacijos architektūra?
  }{exam:question:34}
  <++>
\end{question}

\begin{question}{%
  Informacijos architektūros projektavimo stiliai.
  }{exam:question:35}
  <++>
\end{question}

\begin{question}{%
  Gestalt principai.
  }{exam:question:36}
  <++>
\end{question}

\begin{question}{%
  Ką akcentuoja harmoninga žmogaus ir kompiuterio sąveika?
  }{exam:question:37}
  <++>
\end{question}

\begin{question}{%
  Išvardinkite ir paaiškinkite bent septynis harmoningos sąveikos
  principus.
  }{exam:question:38}
  <++>
\end{question}

\begin{question}{%
  Kurio stiliaus sąsajose panaudojama įgimta žmogaus savybė – erdvinė
  orientacija?
  }{exam:question:39}
  <++>
\end{question}

\begin{question}{%
  Išvardinkite tipinius duomenų vizualizavimo būdus ir užduotis.
  }{exam:question:40}
  <++>
\end{question}

\begin{question}{%
  Palyginkite tiesioginių ir netiesioginių nurodymo įrenginių privalumus
  ir trūkumus.
  }{exam:question:41}
  <++>
\end{question}

\begin{question}{%
  Kuris yra tiesioginio nurodymo įrenginys:
  \begin{itemize}
    \item jutiminis ekranas;
    \item grafinė planšetė;
    \item pelė;
    \item vairasvirtė
  \end{itemize}
  ?
  }{exam:question:42}
  <++>
\end{question}

\begin{question}{%
  Išvardinkite bent penkis novatoriškus sąveikos įrenginius ir
  paaiškinkite jų paskirtį.
  }{exam:question:44}
  <++>
\end{question}

\begin{question}{%
  Kuris klaviatūros išdėstymas pritaikytas naujokui, o kuris vidutiniškai
  patyrusiam naudotojui?
  }{exam:question:45}
  <++>
\end{question}

\begin{question}{%
  Sąsajų kūrimo priemonių lygiai.
  }{exam:question:46}
  <++>
\end{question}

\begin{question}{%
  Sąsajų kūrimo priemonių pasirinkimo kriterijai.
  }{exam:question:47}
  <++>
\end{question}

Nuoroda į 23 klausimą: \ref{exam:question:23}
