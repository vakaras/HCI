

\begin{question}{%
  Kaip nustatomi testavimo metu aptiktų klaidų prioritetai?
  }{exam:question:31}
  <++>
\end{question}

\begin{question}{%
  Paaiškinkite netiesioginius vertinimo su naudotojais būdus.
  }{exam:question:32}
  <++>
\end{question}

\begin{question}{%
  Interneto svetainių euristikos.
  }{exam:question:33}
  <++>
\end{question}

\begin{question}{%
  Kas yra informacijos architektūra?
  }{exam:question:34}
  <++>
\end{question}

\begin{question}{%
  Informacijos architektūros projektavimo stiliai.
  }{exam:question:35}
  <++>
\end{question}

\begin{question}{%
  Gestalt principai.
  }{exam:question:36}
  <++>
\end{question}

\begin{question}{%
  Ką akcentuoja harmoninga žmogaus ir kompiuterio sąveika?
  }{exam:question:37}
  <++>
\end{question}

\begin{question}{%
  Išvardinkite ir paaiškinkite bent septynis harmoningos sąveikos
  principus.
  }{exam:question:38}
  <++>
\end{question}

\begin{question}{%
  Kurio stiliaus sąsajose panaudojama įgimta žmogaus savybė – erdvinė
  orientacija?
  }{exam:question:39}
  <++>
\end{question}

\begin{question}{%
  Išvardinkite tipinius duomenų vizualizavimo būdus ir užduotis.
  }{exam:question:40}
  <++>
\end{question}
