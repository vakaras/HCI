\begin{question}{%
  Kaip nustatomi testavimo metu aptiktų klaidų prioritetai?
  }{exam:question:31}
  Žr.: \cite[20--22]{skaidres-10}
\end{question}

\begin{question}{%
  Paaiškinkite netiesioginius vertinimo su naudotojais būdus.
  }{exam:question:32}

  Protokolavimą atlieka protokolavimo priemonės\cite[37]{skaidres-10}:
  \begin{itemize}
    \item kai kurios fiksuoja klavišų paspaudimo laiką;
    \item kitos įrašo naudotojo ir sistemos sąveiką.
  \end{itemize}
  Naudojant protokolavimą paprastai yra siekiama
  išsiaiškinti\cite[38]{skaidres-10}:
  \begin{itemize}
    \item sistemos funkcinių galimybių naudojimo dažnį;
    \item naudotojų sąsajos objektų naudojimo statistiką.
  \end{itemize}
  Protokolas parodo ką naudotojas veikė, bet neparodo kodėl jis veikė
  būtent tokiu būdu.

  Įvykių dienoraščiuose\cite[49]{skaidres-10} naudotojai yra kviečiami
  protokoluoti savo veiksmus ir pastebėjimus naudojant gaminį.
  Naudingiausia naudoti pristačius jau veikiantį prototipą.

  Savybių sąrašas \cite[56]{skaidres-10} naudojamas projekto viduryje
  testuojant beta versijas, taip pat jau atidavus gaminį užsakovui,
  yra renkama informacija galimiems atnaujinimams ir tobulinimams.
  Pavyzdžiui, juo galima rinkti duomenis, ar po numatyto laikotarpio
  gaminio naudojimas perėjo į pilno naudojimo etapą. (Ar naudotojai
  nurodė, kad žino visas sistemos savybes.)

  Dar yra apklausos ir klausimynai \cite[62]{skaidres-10}. Klausimai
  gali būti:
  \begin{itemize}
    \item atviri;
    \item uždari;
    \item vertinantys (Liberto skalė nuo griežtai nesutinku iki griežtai
      sutinku).
  \end{itemize}

  Per pokalbius \cite[64]{skaidres-10} yra laisvai dalijamasi gaminio
  naudojimo įspūdžiais.
\end{question}

\begin{question}{%
  Interneto svetainių euristikos.
  }{exam:question:33}
  \begin{itemize}
    \item Kokybiškas turinys \en{high quality content}.
    \item Nuolat atnaujinamas \en{often updated}.
    \item Minimalus pakrovimo laikas \en{minimal download time}.
    \item Lengvai naudojamas \en{easy of use}.
    \item Atitikimas naudotojų poreikiams \en{relevant to user needs}.
    \item Kūrmas specialiai tinklalapiui \en{unique to the online medium}.
    \item Tinklinė organizacijos kultūra \en{Netcentric corporate
      culture}.
  \end{itemize}
  Žr.: \cite[37]{skaidres-13}.
\end{question}

\begin{question}{%
  Kas yra informacijos architektūra?
  }{exam:question:34}

  Informacijos architektūra\cite[11]{skaidres-14} – programų sistemų
  ar svetainių struktūros ir funkcinių galimybių organizavimas, siekiant
  palengvinti intuityvią paiešką.

  \begin{enumerate}
    \item Bendrų informacinių aplinkų struktūrinis projektavimas.
    \item Organizavimo, ženklinimo, paieškos ir navigacijos sistemų
      integravimas.
    \item Panaudojamų ir lengvai paieškomų informacinių produktų
      ir patirčių formavimas.
    \item Metodika, skirta pateikti projektavimo ir architektūros
      principus.
  \end{enumerate}
  Žr.: \cite[8]{skaidres-14}.

  Kas nėra informacijos architektūra\cite[12]{skaidres-14}:
  \begin{itemize}
    \item grafinis dizainas;
    \item programinės įrangos kūrimas;
    \item panaudojamumo inžinerija.
  \end{itemize}
\end{question}

\begin{question}{%
  Informacijos architektūros projektavimo stiliai.
  }{exam:question:35}

  Iš viršaus į apačią – prieš apibrėžiant svetainės struktūrą formuluojama:
  \begin{enumerate}
    \item verslo strategija;
    \item naudotojų tikslai;
    \item aukšto lygio svetainės struktūra.
  \end{enumerate}
  Projektuojant šiuo būdu yra siekiama pateikti atsakymus į naudotojui
  rūpimus klausimus:
  \begin{itemize}
    \item kur aš esu?
    \item kaip susirasti tai ko noriu?
    \item kaip naviguoti svetainėje?
    \item kas svarbaus / unikalaus šioje organizacijoje?
    \item kas čia vyksta?
    \item kokia informacija prieinama šioje svetainėje?
    \item kaip galima susisiekti su svetainės šeimininkais?
  \end{itemize}

  Iš apačios į viršų:
  \begin{enumerate}
    \item konkrečių turinio ryšių projektavimas, realizuojant konkrečius
      informacijos paieškos scenarijus;
    \item aukštesnio lygio hierarchijos formavimas.
  \end{enumerate}
  Architektūra iš apačios aukštyn dažniausiai pasireiškia per patį
  turinį, tai yra jo struktūrą ir išdėstymą. Žymės, rikiavimas ir
  struktūra atsako naudotojui į šiuos klausimus:
  \begin{itemize}
    \item kur aš esu?
    \item kas čia yra?
    \item kur dar iš čia galiu nukakti?
  \end{itemize}
  Naudotojai dažnai praleidžia iš viršaus į apačią IA, kadangi iš
  apačios į viršų IA yra intuityvesnė.

  Žr.: \cite[23]{skaidres-14}
\end{question}

\begin{question}{%
  Gestalt principai.
  }{exam:question:36}
  \begin{itemize}
    \item Artumas \en{proximity}.
    \item Tolydumas \en{continuity}.
    \item Visumos dalis \en{part-whole}.
    \item Panašumas \en{similarity}.
    \item Užbaigtumas \en{closure} ir pratęsimas.
    \item Paprastumas.
    \item Simetrija.
    \item Lygiagretumas.
  \end{itemize}
\end{question}

\begin{question}{%
  Ką akcentuoja harmoninga žmogaus ir kompiuterio sąveika?
  }{exam:question:37}
  
  Žmogus dirbdamas turi būti tėkmės būsenoje \cite[5]{skaidres-15}.
  Jos savybės:
  \begin{itemize}
    \item laiko ir aplinkos pojūčių netekimas;
    \item visiška koncentracija ties veiklos tikslu;
    \item gilaus susimąstymo ir meditacijos būsena.
  \end{itemize}
  Rezultatai:
  \begin{itemize}
    \item ekstremalus produktyvumas;
    \item pasitenkinimas.
  \end{itemize}
  Pasiekiama naudojant permatomą sąsają:
  \begin{itemize}
    \item dingsta riba tarp programos ir naudotojo;
    \item visi objektai nukreipti vienam tikslui pasiekti.
  \end{itemize}

\end{question}

\begin{question}{%
  Išvardinkite ir paaiškinkite bent septynis harmoningos sąveikos
  principus.
  }{exam:question:38}
  \begin{itemize}
    \item Mažiau yra daugiau – pavyzdžiui, Google paieška prie senąsias
      paieškos sistemas.
    \item Leisti naudotojui dirbti, neklausinėti – „nemėtyti“ nereikalingų
      langų.
    \item Įrankiai po ranka – reikiami įrankiai lengvai randami,
      dizainas nėra perkrautas.
    \item Tinkamas atsakas \en{feedback} – paprasta ir lengvai prieinama
      būsenos informacija. (Pavyzdys su „Word Count“ „MS Word 2007“
      programoje.)
    \item Tikėtina – galima. Pavyzdžiui, programuotojo logika:
      \begin{equation*}
        \underbrace{1 \land 1 \land \cdots \land 1}_{1000} \land 0 = 0.
      \end{equation*}
      Realiai, kiek dažnai neišsaugome atliktų pakeitimų?
    \item Suprantama informacija: skritulinės diagramos vietoj skaičių
      nurodant kietojo disko vietos panaudojimą.
    \item Įvesti pasirenkant – pavyzdžiui, leisti pasirinkti vieną iš
      standartinių paraščių variantų, vietoj to, kad kiekvieną kartą
      liepti juos nurodyti.
    \item Pateikti objektų ir programos būseną.
    \item Vengti nereikalingų pranešimų – pavyzdžiui, kai spausdintuvas
      spausdina nederėtų rodyti jo kasečių būsenos, ypač jei jos yra
      beveik pilnos, nes naudotojas gali priprasti ignoruoti tą
      pranešimą ir nepastebės, kai tai bus svarbu.
    \item Vengti tuščių vietų.
    \item Skirti komandas ir konfigūravimą.
    \item Leisti pasirinkimus.
    \item Paslėpti „katapultas“ – nederėtų įtaisyti katapultavimosi mygtuko
      šalia radijo ir kabinos apšvietimo, nes tada jį gali nuspausti
      tiesiog netyčia… Visgi jis turėtų būti pasiekiamas, kai jo reikia.
    \item Optimizuoti atsakus – sistema turi sureaguoti į naudotojo
      veiksmus per kuo mažesnį laiko tarpą.
  \end{itemize}
  Žr.: \cite[6--21]{skaidres-15}
\end{question}

\begin{question}{%
  Kurio stiliaus sąsajose panaudojama įgimta žmogaus savybė – erdvinė
  orientacija?
  }{exam:question:39}
  <++>
  FIXME: Virtualiųjų pasaulių? Darbastalio\cite[38]{skaidres-15}, 3D
  aplinkose?
\end{question}

\begin{question}{%
  Išvardinkite tipinius duomenų vizualizavimo būdus ir užduotis.
  }{exam:question:40}

  FIXME: Kokie yra tipiniai vizualizavimo būdai?
  Gal: \cite[50--64]{skaidres-15}

  Informacijos tipai\cite[49]{skaidres-15}:
  \begin{enumerate}
    \item vienmatė;
    \item dvimatė;
    \item trimatė;
    \item daugiamatė;
    \item laikinė;
    \item hierarchinė;
    \item tinklinė.
  \end{enumerate}

  Užduotys\cite[49]{skaidres-15}:
  \begin{itemize}
    \item peržvelgti;
    \item priartinti;
    \item filtruoti;
    \item pateikti detales;
    \item parodyti santykį;
    \item rodyti retrospektyvą;
    \item išrinkti.
  \end{itemize}

\end{question}
