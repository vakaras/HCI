\begin{question}{%
  Kaip nustatomi testavimo metu aptiktų klaidų prioritetai?
  }{exam:question:31}
  Žr.: \cite[20--22]{skaidres-10}
\end{question}

\begin{question}{%
  Paaiškinkite netiesioginius vertinimo su naudotojais būdus.
  }{exam:question:32}

  Protokolavimą atlieka protokolavimo priemonės\cite[37]{skaidres-10}:
  \begin{itemize}
    \item kai kurios fiksuoja klavišų paspaudimo laiką;
    \item kitos įrašo naudotojo ir sistemos sąveiką.
  \end{itemize}
  Naudojant protokolavimą paprastai yra siekiama
  išsiaiškinti\cite[38]{skaidres-10}:
  \begin{itemize}
    \item sistemos funkcinių galimybių naudojimo dažnį;
    \item naudotojų sąsajos objektų naudojimo statistiką.
  \end{itemize}
  Protokolas parodo ką naudotojas veikė, bet neparodo kodėl jis veikė
  būtent tokiu būdu.

  Įvykių dienoraščiuose\cite[49]{skaidres-10} naudotojai yra kviečiami
  protokoluoti savo veiksmus ir pastebėjimus naudojant gaminį.
  Naudingiausia naudoti pristačius jau veikiantį prototipą.

  Savybių sąrašas \cite[56]{skaidres-10} naudojamas projekto viduryje
  testuojant beta versijas, taip pat jau atidavus gaminį užsakovui,
  yra renkama informacija galimiems atnaujinimams ir tobulinimams.
  Pavyzdžiui, juo galima rinkti duomenis, ar po numatyto laikotarpio
  gaminio naudojimas perėjo į pilno naudojimo etapą. (Ar naudotojai
  nurodė, kad žino visas sistemos savybes.)

  Dar yra apklausos ir klausimynai \cite[62]{skaidres-10}. Klausimai
  gali būti:
  \begin{itemize}
    \item atviri;
    \item uždari;
    \item vertinantys (Liberto skalė nuo griežtai nesutinku iki griežtai
      sutinku).
  \end{itemize}

  Per pokalbius \cite[64]{skaidres-10} yra laisvai dalijamasi gaminio
  naudojimo įspūdžiais.
\end{question}

\begin{question}{%
  Interneto svetainių euristikos.
  }{exam:question:33}
  \begin{itemize}
    \item Kokybiškas turinys \en{high quality content}.
    \item Nuolat atnaujinamas \en{often updated}.
    \item Minimalus pakrovimo laikas \en{minimal download time}.
    \item Lengvai naudojamas \en{easy of use}.
    \item Atitikimas naudotojų poreikiams \en{relevant to user needs}.
    \item Kūrmas specialiai tinklalapiui \en{unique to the online medium}.
    \item Tinklinė organizacijos kultūra \en{Netcentric corporate
      culture}.
  \end{itemize}
  Žr.: \cite[37]{skaidres-13}.
\end{question}

\begin{question}{%
  Kas yra informacijos architektūra?
  }{exam:question:34}
  <++>
\end{question}

\begin{question}{%
  Informacijos architektūros projektavimo stiliai.
  }{exam:question:35}
  <++>
\end{question}

\begin{question}{%
  Gestalt principai.
  }{exam:question:36}
  <++>
\end{question}

\begin{question}{%
  Ką akcentuoja harmoninga žmogaus ir kompiuterio sąveika?
  }{exam:question:37}
  <++>
\end{question}

\begin{question}{%
  Išvardinkite ir paaiškinkite bent septynis harmoningos sąveikos
  principus.
  }{exam:question:38}
  <++>
\end{question}

\begin{question}{%
  Kurio stiliaus sąsajose panaudojama įgimta žmogaus savybė – erdvinė
  orientacija?
  }{exam:question:39}
  <++>
\end{question}

\begin{question}{%
  Išvardinkite tipinius duomenų vizualizavimo būdus ir užduotis.
  }{exam:question:40}
  <++>
\end{question}
