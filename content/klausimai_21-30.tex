\begin{question}{%
  Tarkime, turite sąsajos maketą ir galite pasirinkti \emph{tik vieną}
  iš žemiau pateiktų metodų:
  \begin{itemize}
    \item euristinis tikrinimas;
    \item pažintinė peržvalga;
    \item testavimas su naudotojais.
  \end{itemize}
  Argumentuokite savo pasirinkimą.
  }{exam:question:21}

  <++>

  Pažintinė peržvalga\cite[13]{skaidres-9}:
  \begin{itemize}
    \item tikslus užduočių loginių žingsnių tikrinimas;
    \item naudojama projekto pradžioje, kai atsiranda eskizinis (paprastai
      popierinis) sąsajos maketas;
    \item paremta solidžia pažintinės psichologijos teorija.
  \end{itemize}
  Ji taikoma\cite[56]{skaidres-9}:
  \begin{itemize}
    \item eskiziniams ir detaliesiems maketams;
    \item sudėtingesnėms esminėms užduotims;
    \item riboto naudojimo etapo efektyvumo tikslams vertinti.
  \end{itemize}
  
  Ekspertų tikrinimai\cite[9]{skaidres-8}:
  \begin{itemize}
    \item paremti pažintinės psichologijos modeliais arba projektavimo
      euristikomis;
    \item leidžia vertinti idėjas nuo projekto pradžios;
    \item nebūtinas detalusis maketas (taupomas laikas, skirtas maketui
      kurti);
    \item nereikia planuoti ir vykdyti eksperimento (testavimo);
    \item vykdomi panaudojamumo vertintojų ar sąsajos architektų.
  \end{itemize}
\end{question}

\begin{question}{%
  Tarkime, turite interaktyvų prototipą ir galite pasirinkti \emph{tik
  vieną} ir žemiau pateiktų metodų:
  \begin{itemize}
    \item euristinis tikrinimas;
    \item pažintinė peržvalga;
    \item testavimas su naudotojais.
  \end{itemize}
  Argumentuokite savo pasirinkimą.
  }{exam:question:22}

  Turint interaktyvų prototipą galima efektyviai vykdyti testavimą
  su naudotojais. Jį ir reikėtų rinktis, nes jis, kitaip nei 
  euristinis tikrinimas ar pažintinė peržvalga, leidžia patikrinti
  ar kūrėjai teisingai supranta būsimų naudotojų poreikius.

  Panaudojamumo testavimu yra vertinamas\cite[8]{skaidres-8}:
  \begin{itemize}
    \item sąsajos bandomasis pavyzdys (pakankamai detalus maketas) arba
    \item veikianti programų sistema.
  \end{itemize}
\end{question}

\begin{question}{%
  Kuriuose vertinimo metoduose vertintojas bando numatyti, ar naujokas
  lengvai išmoks naudotis sistema:
  \begin{itemize}
    \item euristiniame tikrinime;
    \item pažintinėje peržvalgoje;
    \item KLM metode;
    \item testavime?
  \end{itemize}
  (Pasirinkite visus tinkančius atsakymus.)
  }{exam:question:23}
  <++>
  Taip:
  \begin{itemize}
    \item testavime.
    \item pažintinėje peržvalgoje – vertinamas sistemos efektyvumas riboto
      naudojimo etape.
  \end{itemize}

  Ne:
  \begin{itemize}
    \item KLM metode – juo yra bandoma įvertinti patyrusio naudotojo
      užduočių atlikimo greitį.
  \end{itemize}
\end{question}

\begin{question}{%
  Paaiškinkite GOMS ir KLM santykį.
  }{exam:question:24}
  
  GOMS modelis\cite[14]{skaidres-8}:
  \begin{itemize}
    \item \emph{goals} – aukšto lygmens naudotojo tikslai;
    \item \emph{operators} – veiksmai tikslui pasiekti;
    \item \emph{methods} – tikslo pasiekimo procedūra;
    \item \emph{selection rules} – procedūros pasirinkimo taisyklės.
  \end{itemize}
  Naudojant GOMS modelį yra identifikuojamos veiksmų sekos
  \cite[15]{skaidres-8}. Iš principo vietoj GOMS aprašo gali būti
  naudojamas eskizinis maketas arba hierarchinės užduočių analizės
  aprašas (bet tuo atveju jis turi būti detalizuotas iki primityvių
  veiksmų).

  KLM (Keystroke-Level Model) leidžia prognozuoti užduoties atliekamos
  be klaidų įvykdymo laiką. 

  Veiksmų analizės tikslas yra numatyti patyrusio naudotojo veiklos
  našumą. GOMS yra būdas dekomponuoti užduotį iki elementarių veiksmų,
  o tada su KLM galime apskaičiuoti užduoties įvykdymo laiką esant
  prielaidoms\cite[40]{skaidres-8}:
  \begin{itemize}
    \item naudotojas moka atlikti užduotį;
    \item naudotojas atlieka užduotį be klaidų.
  \end{itemize}
  Tokiu būdu galime įvertinti užduočių atlikimo našumą:
  \begin{itemize}
    \item riboto naudojimo etapui – naudotojas nežino arba nenaudoja
      santrumpų;
    \item pilno naudojimo etapui – našiausio scenarijaus trukmė.
  \end{itemize}
\end{question}

\begin{question}{%
  Kas yra ir kokio naudotojo tipo veikimą tikrina veiksmų analizės metodai?
  }{exam:question:25}
  Žr.: \ref{exam:question:24}.
\end{question}

\begin{question}{%
  Palyginkite panaudojamumo vertinimus: modelinius ir ekspertų apžvalgas,
  konkrečius metodus, pavyzdžiui, pažintinę peržvalgą, euristinį
  tikrinimą, testavimą su naudotojais šiais aspektais:
  \begin{itemize}
    \item paskirtis,
    \item kokiuose projekto etapuose tikslinga vykdyti,
    \item vertinimo dalyviai,
    \item ką reikia paruošti vertinimui,
    \item kas yra vertinimo rezultatas
  \end{itemize}
  ?
  }{exam:question:26}
  <++>
  Modeliniai vertinimai: nuo \cite[42]{skaidres-8}.

  Pažintinė peržvalga:
  \begin{itemize}
    \item paskirtis: įvertinti riboto naudojimo užduočių atlikimo
      efektyvumą (santykį informacijos, kurios reikia užduoties
      atlikimui, su informacija kuri yra pateikiama sąsajos);
    \item tikslinga vykdyti: projekto pradžioje, kai turima tik eskizinį
      (paprastai popierinį) sąsajos maketą;
    \item vertinimo dalyviai: panaudojamumo ekspertai, kartais gali būti
      ir programuotojai;
    \item vertinimo įvestis:
      \begin{itemize}
        \item naudotojų ir užduočių charakteristikos;
        \item konkrečių naudojimo scenarijų aibė;
        \item sąsajos maketas – turi būti matomas langų išdėstymas ir
          aiški veiksmų tvarka;
      \end{itemize}
    \item rezultatas: pažintinio apėjimo ataskaita ir iš jos gautas
      defektų sąrašas.
  \end{itemize}

  Testavimas su naudotojais:
  \begin{itemize}
    \item paskirtis:
    \item tikslinga vykdyti:
    \item vertinimo dalyviai: būsimi naudotojai;
    \item reikia paruošti: užduotis, sutikimus, klausimynus, įrašymo
      įrangą \cite[13]{skaidres-10};
    \item vertinimo rezultatas: ataskaita\cite[27--29]{skaidres-10}
  \end{itemize}

  Euristinis tikrinimas\cite{skaidres-11}:
  \begin{itemize}
    \item paskirtinis:
    \item tikslinga vykdyti:
    \item vertinimo dalyviai: ekspertai (specialistai geriau suprantantys
      naudotojus, nei programuotojai);
    \item reikia paruošti: 
    \item vertinimo rezultatas:
  \end{itemize}
\end{question}

\begin{question}{%
  Ką teigia Fito dėsnis? (Paaiškinkite esmę, formulės pateikti nereikia.)
  }{exam:question:27}

  \en{Fitt's Law}

  Kuo didesnis yra objekto ir kursoriaus atstumas arba kuo mažesnis
  yra objekto dydis, tuo naudotojui reikia daugiau laiko, kad jį
  pažymėtų.

  \begin{equation*}
    \t{Laikas (ms)} = a + b \log_2 \left( \frac{D}{S} + 1 \right),
  \end{equation*}
  čia:
  \begin{description}
    \item[$D$] – atstumas nuo pelės iki objekto;
    \item[$S$] – objekto plotis.
  \end{description}

  Žr.: \cite[48]{skaidres-8}.
\end{question}

\begin{question}{%
  Ką teigia Hiko dėsnis? (Paaiškinkite esmę, formulės pateikti nereikia.)
  }{exam:question:28}

  \en{Hick's Law}

  Kuo didesnis vieno tipo pasirinkimų skaičius, tuo daugiau laiko
  reikia pasirinkimui.

  \begin{equation*}
    \t{Laikas (ms)} = k \log_2 \left( n + 1 \right),
  \end{equation*}
  čia $k \approx 150$ ms.

  Žr.: \cite[51]{skaidres-8}
\end{question}

\begin{question}{%
  Palyginkite darbalaukinių ir kitokio tipo sistemų euristikas.
  }{exam:question:29}
  <++>
\end{question}

\begin{question}{%
  Kokios euristikos aktualios:
  \begin{itemize}
    \item naujokui,
    \item vidutiniškai patyrusiam,
    \item patyrusiam
  \end{itemize}
  naudotojui?
  }{exam:question:30}
  <++>
\end{question}
