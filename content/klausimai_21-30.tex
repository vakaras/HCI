\begin{question}{%
  Tarkime, turite sąsajos maketą ir galite pasirinkti \emph{tik vieną}
  iš žemiau pateiktų metodų:
  \begin{itemize}
    \item euristinis tikrinimas;
    \item pažintinė peržvalga;
    \item testavimas su naudotojais.
  \end{itemize}
  Argumentuokite savo pasirinkimą.
  }{exam:question:21}
  <++>
\end{question}

\begin{question}{%
  Tarkime, turite interaktyvų prototipą ir galite pasirinkti \emph{tik
  vieną} ir žemiau pateiktų metodų:
  \begin{itemize}
    \item euristinis tikrinimas;
    \item pažintinė peržvalga;
    \item testavimas su naudotojais.
  \end{itemize}
  Argumentuokite savo pasirinkimą.
  }{exam:question:22}
  <++>
\end{question}

\begin{question}{%
  Kuriuose vertinimo metoduose vertintojas bando numatyti, ar naujokas
  lengvai išmoks naudotis sistema:
  \begin{itemize}
    \item euristiniame tikrinime;
    \item pažintinėje peržvalgoje;
    \item KLM metode;
    \item testavime?
  \end{itemize}
  (Pasirinkite visus tinkančius atsakymus.)
  }{exam:question:23}
  <++>
\end{question}

\begin{question}{%
  Paaiškinkite GOMS ir KLM santykį.
  }{exam:question:24}
  <++>
\end{question}

\begin{question}{%
  Kas yra ir kokio naudotojo tipo veikimą tikrina veiksmų analizės metodai?
  }{exam:question:25}
  <++>
\end{question}

\begin{question}{%
  Palyginkite panaudojamumo vertinimus: modelinius ir ekspertų apžvalgas,
  konkrečius metodus, pavyzdžiui, pažintinę peržvalgą, euristinį
  tikrinimą, testavimą su naudotojais šiais aspektais:
  \begin{itemize}
    \item paskirtis,
    \item kokiuose projekto etapuose tikslinga vykdyti,
    \item vertinimo dalyviai,
    \item ką reikia paruošti vertinimui,
    \item kas yra vertinimo rezultatas
  \end{itemize}
  ?
  }{exam:question:26}
  <++>
\end{question}

\begin{question}{%
  Ką teigia Fito dėsnis? (Paaiškinkite esmę, formulės pateikti nereikia.)
  }{exam:question:27}
  <++>
\end{question}

\begin{question}{%
  Ką teigia Hiko dėsnis? (Paaiškinkite esmę, formulės pateikti nereikia.)
  }{exam:question:28}
  <++>
\end{question}

\begin{question}{%
  Palyginkite darbalaukinių ir kitokio tipo sistemų euristikas.
  }{exam:question:29}
  <++>
\end{question}

\begin{question}{%
  Kokios euristikos aktualios:
  \begin{itemize}
    \item naujokui,
    \item vidutiniškai patyrusiam,
    \item patyrusiam
  \end{itemize}
  naudotojui?
  }{exam:question:30}
  <++>
\end{question}
