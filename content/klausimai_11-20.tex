\begin{question}{%
  Naudotojų tipų poreikiai.
  }{exam:question:11}
  <++>
\end{question}

\begin{question}{%
  Personų paskirtis tiksliniame projektavime. Personos aprašo struktūra.
  }{exam:question:12}
  <++>
\end{question}

\begin{question}{%
  Palyginkite naudotojo potyrių ir panaudojamumo tikslus.
  }{exam:question:13}
  <++>
\end{question}

\begin{question}{%
  Naudotojui palankus projektavimas.
  }{exam:question:14}
  <++>
\end{question}

\begin{question}{%
  Hierarchinė užduočių analizė.
  }{exam:question:15}
  <++>
\end{question}

\begin{question}{%
  Maketavimo paskirtis.
  }{exam:question:16}
  <++>
\end{question}

\begin{question}{%
  Paaiškinkite Gareto naudotojo potyrių projektavimo sluoksnius.
  }{exam:question:17}
  <++>
\end{question}

\begin{question}{%
  Kaip naudotojui palankus projektavimas vykdomas stambiuose projektuose?
  Pateikite atitinkamų metodikų pavyzdžius.
  }{exam:question:18}
  <++>
\end{question}

\begin{question}{%
  Paaiškinkite Normano sąveikos etapų modelį ir indėlį sąsajos
  projektavimui.
  }{exam:question:19}
  <++>
\end{question}

\begin{question}{%
  Keturi gero projektavimo principai išplaukiantys ir Normano sąveikos
  etapo modelio.
  }{exam:question:20}
  <++>
\end{question}
