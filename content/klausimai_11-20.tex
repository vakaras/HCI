\begin{question}{%
  Sėkmingo produkto savybės.
  }{exam:question:09}
  FIXME: Būtina praplėsti.
  \begin{itemize}
    \item Patrauklumas.
    \item Įgyvendinamumas.
    \item Veiksnumas.
  \end{itemize}
  Žr.: \cite[5]{skaidres-5}.
\end{question}

\begin{question}{%
  Tikslinio projektavimo etapai.
  }{exam:question:10}

  Tikslinis projektavimas\cite[54]{skaidres-5} – panaudojamumo principams
  suteikiami prioritetai atsižvelgiant į naudotojo poreikius.

  \begin{enumerate}
    \item \emph{Analizė} naudotojai, dalykinė sritis.
    \item \emph{Modeliavimas} personos, veiklų srautai, artefaktai.
    \item \emph{Reikalavimai} naudotojų, verslo, techniniai poreikiai.
    \item \emph{Eskizai} sąveikos struktūra ir veiklų srautai.
    \item \emph{Detalieji prototipai} sąsaj̇ų detalizavimas, turinio
      išbaigimas.
    \item \emph{Parama} realizacijos parama ir priežiūra.
  \end{enumerate}
  Žr.: \cite[6]{skaidres-5}.
\end{question}

\begin{question}{%
  Naudotojų tipų poreikiai.
  }{exam:question:11}
  Žr.: \cite[11--]{skaidres-5}.
\end{question}

\begin{question}{%
  Personų paskirtis tiksliniame projektavime. Personos aprašo struktūra.
  }{exam:question:12}
  Personos esmė yra aprašyti naudotojo veiklų tikslus ir lūkesčius. Jos yra 
  skirtos:
  \begin{itemize}
    \item nustatyti gaminio funkcijas ir elgseną;
    \item komunikuoti su suinteresuotais asmenimis;
    \item ieškoti kompromisų;
    \item matuoti projekto efektyvumą;
    \item padėti kitoms projekto veikloms (pavyzdžiui, rinkodaros skyriui);
    \item pritaikyti produktą įvairioms grupėms. (Vidurkinis
      naudotojas neegzistuoja!)
  \end{itemize}
  Žr.: \cite[39--47]{skaidres-5}.
  Personos apraše yra nurodoma\cite[49]{skaidres-5}:
  \begin{itemize}
    \item pagrindinė demografinė informacija (amžius, išsilavinimas,
      pareigos, lytis);
    \item siekiai konkrečiame projekte;
    \item charakteristikos:
      \begin{itemize}
        \item kokiomis informacinių technologijų priemonėmis naudojasi;
        \item motyvacija arba galimybės tobulinti įgūdžius;
        \item prieinama parama (pavyzdžiui, jei dirbama kolektyve,
          paprastai yra prieinama „vietinių ekspertų“ parama, nes
          visuomet kažkas yra geriau įvaldęs konkrečią technologiją);
      \end{itemize}
    \item naudotojo tipas (naujokas, vidutiniškai patyręs, ekspertas);
    \item vykdomų ir planuojamų kompiuterizuoti užduočių dažnis ir trukmė;
    \item su kokiomis problemomis susiduria esamoje situacijoje;
    \item patobulintos sąveikos vizija:
      \begin{itemize}
        \item kaip galėtų būti patobulinta esama situacija;
        \item kaip persona pageidautų veikti;
        \item kokių situacijų ar veiklų pageidautų išvengti (tai yra, ką
          pageidautų automatizuoti);
      \end{itemize}
    \item būsimos sistemos esminių užduočių panaudojamumo tikslai.
  \end{itemize}
\end{question}

\begin{question}{%
  Palyginkite naudotojo potyrių ir panaudojamumo tikslus.
  }{exam:question:13}

  Panaudojamumo tikslai\cite[20--]{skaidres-5}:
  \begin{itemize}
    \item našumas:
      \begin{itemize}
        \item kriterijus: laikas, atsakymų skaičius;
        \item lankstumas: alternatyvių būdų skaičius;
      \end{itemize}
    \item efektyvumas:
      \begin{itemize}
        \item išmokstamumas – išmokimo laikas, perėjimas į pilno naudojimo
          etapą;
        \item robastiškumas – užduoties įvykdymas;
      \end{itemize}
    \item malonumas.
  \end{itemize}
  Poreikius formuluoja produkto aplinka: verslo tikslai, naudotojų ypatybės,
  dalykinė sritis.

  Panaudojamumo tikslai\cite[47p.]{konspektas}:
  \begin{itemize}
    \item efektyvumas;
    \item našumas;
    \item saugumas;
    \item naudingumas;
    \item išmokstamumas;
    \item lengvas atsimenamumas.
  \end{itemize}

  \begin{defn}[Naudotojų potyrių projektavimas]
    Žmogaus ir kompiuterio sąveikos dalis, akcentuojanti produkto naudotojo
    patiriamus pojūčius.
    \cite[3]{skaidres-7}
  \end{defn}

  \begin{defn}[Naudotojo potyriai]
    Asmens pojūčiai ir reakcijos, sukelti naudojant arba numatant naudoti
    produktą, sistemą arba paslaugą.\cite[4]{skaidres-7}
  \end{defn}

  Naudotojo potyrių tikslai\cite[3]{skaidres-7}:
  \begin{itemize}
    \item malonumas;
    \item smagumas;
    \item linksmumas;
    \item patrauklumas;
    \item estetinis patrauklumas;
    \item parama kūrybiškumui;
    \item pasitenkinimas.
  \end{itemize}
\end{question}

\begin{question}{%
  Naudotojui palankus projektavimas.
  }{exam:question:14}
  
  Naudotojui palankaus projektavimo esmė – kuo ankstyvesniame etape
  įtraukti naudotoją į projektavimo procesą.\cite[165p.]{konspektas}
  Jis yra iteratyvus ir įtraukiantis naudotojus į projektavimą:
  \begin{itemize}
    \item nagrinėjant personas ir formuluojant joms panaudojamumo
      tikslus;
    \item analizuojant užduotis;
    \item kuriant eskizus;
    \item testuojant juos.
  \end{itemize}

  Projektuojant sąveiką yra atsižvelgiama į\cite[6]{skaidres-6}:
  \begin{itemize}
    \item kas naudos sistemą;
    \item kokioms veikloms (siųsti pranešimus, rinkti informaciją,
      piešti, programuoti, planuoti, žaisti);
    \item kur vyks sąveika.
  \end{itemize}
  Taip par yra siekiama optimizuoti sąveiką, kad ji atitiktų naudotojų
  veiklas ir poreikius.

  Sąveikos projektavimo ciklas\cite[10]{skaidres-6}:
  \begin{enumerate}
    \item identifikuoti poreikius ir nustatyti reikalavimus;
    \item kurti alternatyvius maketus;
    \item kurti interaktyvius maketus;
    \item vertinti ir tobulinti.
  \end{enumerate}

\end{question}

\begin{question}{%
  Hierarchinė užduočių analizė.
  }{exam:question:15}

  Susideda iš dviejų dalių:
  \begin{itemize}
    \item atliekama užduočių dekompozicija;
    \item apibrėžiami použduočių vykdymo planai (tiesiog nurodoma kokia
      tvarka vykdomos použduotys su ciklais, sąlygomis ir t.t.).
  \end{itemize}

  Žr.: \cite[40--43]{skaidres-6}.
\end{question}

\begin{question}{%
  Maketavimo paskirtis.
  }{exam:question:16}

  Maketavimo tikslas yra išbandyti pasirinktą sąveikos rūšį. Svarbu yra,
  kad tai būtų pigu ir greita. Maketuose turi būti matoma sąsajos struktūra
  ir jos metaforos. Maketas yra reikalingas tam, kad galėtume įvertinti
  panaudojamumo tikslus:
  \begin{itemize}
    \item našumą;
    \item efektyvumą;
    \item lankstumą;
    \item robastiškumą.
  \end{itemize}

  Maketas – sistemos modelis, leidžiantis naudotojamas su juo sąveikauti,
  nagrinėti atitikimą poreikiams. Maketai gali būti:
  \begin{itemize}
    \item popieriniai;
    \item naudojimo pavyzdžių piešiniai;
    \item kompiuterinė animacija;
    \item greiti prototipai \en{rapid prototype}.
  \end{itemize}
  
  Detalusis prototipas – dalinai veikianti sistema.

  Maketai nuo detaliųjų prototipų skiriasi resursų sąnaudomis ir detalumu.

  Žr.: \cite[47--]{skaidres-6}
\end{question}


\begin{question}{%
  Paaiškinkite Gareto naudotojo potyrių projektavimo sluoksnius.
  }{exam:question:17}
  
  \begin{itemize}
    \item \emph{Paviršius:} Paviršiaus sluoksnis apibrėžia galutinį
      dizainą.
    \item \emph{Karkasas:} Karkaso sluoksnis apibrėžia ekrano išdėstymą
      ir funkcines lango dalis.
    \item \emph{Struktūra:} Struktūros apibrėžia naudotojo judėjimo
      takus arba navigaciją.
    \item \emph{Apimtis:} Kompiuterizuojamos naudotojo užduotys.
    \item \emph{Strategija:} Strategiją formuoja verslo tikslai, naudotojų
      rolės ir kontekstas.
  \end{itemize}
  Žr.: \cite[9-14]{skaidres-7}.
\end{question}

\begin{question}{%
  Kaip naudotojui palankus projektavimas vykdomas stambiuose projektuose?
  Pateikite atitinkamų metodikų pavyzdžius.
  }{exam:question:18}

  Naudotojui palankių projektavimo metodikų esmė yra kuo ankstyvesniame
  etape įtraukti naudotoją į projektavimo procesą. Naudotojų analizė,
  atlikta ankstyvuosiuose projekto etapuose ženkliai sumažina projektavimo
  laiką ir išlaidas. Tokias sistemas lengviau išmokti, naudoti ir
  prižiūrėti. Programų sistemų metodikos efektyviai remia techninę
  projekto dalį, tačiau jose ne visada yra apibrėžti naudotojo ir jų
  porekių analizės metodai, o kartu ir praktiškų naudotojo sąsajų
  kūrimo procedūros.\cite[165p.]{konspektas}

  IBM „Easy of Use“ etapai:
  \begin{enumerate}
    \item verslo strategija:
      \begin{itemize}
        \item naudotojų patirties analitikas atlieka pradinį naudotojų
          inžinerijos planavimą;
        \item rinkos planuotojas atlieka verslo ir rinkos reikalavimų
          kūrimą;
      \end{itemize}
    \item naudotojų poreikių analizė:
      TODO \cite[39]{skaidres-7}
    \item eskizinis projektavimas:
      TODO \cite[40]{skaidres-7}
    \item detalusis projektavimas:
      TODO \cite[41]{skaidres-7}
    \item diegimas:
      TODO \cite[42]{skaidres-7}
  \end{enumerate}

  Loginės naudotojui palankios interaktyvios projektavimo metodikos
  \en{Logical user-centred interactive design methodology}, sutrumpintai
  LUCID etapai:
  \begin{enumerate}
    \item vizija:
      \begin{itemize}
        \item su visais suinteresuotais asmenimis suderinti bendrą
          aiškią gaminio viziją, išreikštą eskizu;
        \item numatyti potencialias problemas, galinčias sutrukdyti
          kūrimo komandos bendrą veiklą;
      \end{itemize}
    \item naudotojo analizė:
      \begin{itemize}
        \item naudotojų ir užduočių charakteristikos;
        \item kuriami naudotojo mąstymo modeliai;
        \item renkama naudotojų terminija;
        \item surinkti duomenys analizuojami;
        \item panaudojamumo tikslai;
      \end{itemize}
    \item eskizinis projektas:
      \begin{itemize}
        \item koncepcinis projektas;
        \item ekraninis prototipas;
        \item panaudojamumo testavimas;
        \item ekraninio prototipo tobulinimas;
        \item testavimas ir tobulinimas;
      \end{itemize}
    \item detalusis projektas:
      TODO \cite[48]{skaidres-7}
    \item realizacija:
      \begin{itemize}
        \item konsultuoti programuotoją ir, jei reikia, keisti langų
          projektą;
        \item vertinti svarbesnių sąsajos dalių panaudojamumą;
        \item remti programavimo procesą, konsultuojant ir įnešant
          pakeitimus;
      \end{itemize}
    \item išleidimas:
      \begin{itemize}
        \item planuojama gaminio priežiūra;
        \item galutiniai panaudojamumo testavimai naudotojo aplinkoje;
        \item matuoti naudotojų panaudojamumo tikslų pasiekimą;
        \item dokumentuojama projekto metu sukaupta patirtis.
      \end{itemize}
  \end{enumerate}

  Etnografija yra antropologijos šaka, nagrinėjanti ir moksliškai
  aprašinėjanti žmonių bendruomenių kultūras.

  Etnografinių stebėjimų etapai:
  \begin{itemize}
    \item pasiruošimas:
      TODO: \cite[52]{skaidres-7}
    \item tyrinėjimas vietoje:
      TODO: \cite[52]{skaidres-7}
    \item analizė:
      TODO: \cite[53]{skaidres-7}
    \item ataskaitų rengimas:
      TODO: \cite[53]{skaidres-7}
  \end{itemize}
  Etnografiniai stebėjimai iš esmės svarbūs tuo, kad remiasi etnografų
  sukurta ir per ilgą laiką patikrinta stebėjimo metodika, dėl kurios
  naudojimo sumažėja tikimybė susimauti.

  Projektavimas kartu \en{Participatory Design}:
  TODO: \cite[55--56]{skaidres-7}

  Senarijų kūrimas. TODO: \cite[57]{skaidres-7}

  Socialinio poveikio ataskaita ankstyvai projektavimo apžvalgai.
  TODO: \cite[58]{skaidres-7}

  Žr.: \cite[36--]{skaidres-7} ir \cite[165--]{konspektas}.
\end{question}

\begin{question}{%
  Paaiškinkite Normano sąveikos etapų modelį ir indėlį sąsajos
  projektavimui.
  }{exam:question:19}
  <++>
\end{question}

\begin{question}{%
  Keturi gero projektavimo principai išplaukiantys ir Normano sąveikos
  etapo modelio.
  }{exam:question:20}
  <++>
\end{question}
