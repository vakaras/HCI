\begin{question}{%
  Sėkmingo produkto savybės.
  }{exam:question:09}
  FIXME: Būtina praplėsti.
  \begin{itemize}
    \item Patrauklumas.
    \item Įgyvendinamumas.
    \item Veiksnumas.
  \end{itemize}
  Žr.: \cite[5]{skaidres-5}.
\end{question}

\begin{question}{%
  Tikslinio projektavimo etapai.
  }{exam:question:10}

  Tikslinis projektavimas\cite[54]{skaidres-5} – panaudojamumo principams
  suteikiami prioritetai atsižvelgiant į naudotojo poreikius.

  \begin{enumerate}
    \item \emph{Analizė} naudotojai, dalykinė sritis.
    \item \emph{Modeliavimas} personos, veiklų srautai, artefaktai.
    \item \emph{Reikalavimai} naudotojų, verslo, techniniai poreikiai.
    \item \emph{Eskizai} sąveikos struktūra ir veiklų srautai.
    \item \emph{Detalieji prototipai} sąsaj̇ų detalizavimas, turinio
      išbaigimas.
    \item \emph{Parama} realizacijos parama ir priežiūra.
  \end{enumerate}
  Žr.: \cite[6]{skaidres-5}.
\end{question}

\begin{question}{%
  Naudotojų tipų poreikiai.
  }{exam:question:11}
  Žr.: \cite[11--]{skaidres-5}.
\end{question}

\begin{question}{%
  Personų paskirtis tiksliniame projektavime. Personos aprašo struktūra.
  }{exam:question:12}
  Personos esmė yra aprašyti naudotojo veiklų tikslus ir lūkesčius. Jos yra 
  skirtos:
  \begin{itemize}
    \item nustatyti gaminio funkcijas ir elgseną;
    \item komunikuoti su suinteresuotais asmenimis;
    \item ieškoti kompromisų;
    \item matuoti projekto efektyvumą;
    \item padėti kitoms projekto veikloms (pavyzdžiui, rinkodaros skyriui);
    \item pritaikyti produktą įvairioms grupėms. (Vidurkinis
      naudotojas neegzistuoja!)
  \end{itemize}
  Žr.: \cite[39--47]{skaidres-5}.
  Personos apraše yra nurodoma\cite[49]{skaidres-5}:
  \begin{itemize}
    \item pagrindinė demografinė informacija (amžius, išsilavinimas,
      pareigos, lytis);
    \item siekiai konkrečiame projekte;
    \item charakteristikos:
      \begin{itemize}
        \item kokiomis informacinių technologijų priemonėmis naudojasi;
        \item motyvacija arba galimybės tobulinti įgūdžius;
        \item prieinama parama (pavyzdžiui, jei dirbama kolektyve,
          paprastai yra prieinama „vietinių ekspertų“ parama, nes
          visuomet kažkas yra geriau įvaldęs konkrečią technologiją);
      \end{itemize}
    \item naudotojo tipas (naujokas, vidutiniškai patyręs, ekspertas);
    \item vykdomų ir planuojamų kompiuterizuoti užduočių dažnis ir trukmė;
    \item su kokiomis problemomis susiduria esamoje situacijoje;
    \item patobulintos sąveikos vizija:
      \begin{itemize}
        \item kaip galėtų būti patobulinta esama situacija;
        \item kaip persona pageidautų veikti;
        \item kokių situacijų ar veiklų pageidautų išvengti (tai yra, ką
          pageidautų automatizuoti);
      \end{itemize}
    \item būsimos sistemos esminių užduočių panaudojamumo tikslai.
  \end{itemize}
\end{question}

\begin{question}{%
  Palyginkite naudotojo potyrių ir panaudojamumo tikslus.
  }{exam:question:13}

  FIXME: Kas yra naudotojo potyrių tikslai?

  Panaudojamumo tikslai\cite[20--]{skaidres-5}:
  \begin{itemize}
    \item našumas:
      \begin{itemize}
        \item kriterijus: laikas, atsakymų skaičius;
        \item lankstumas: alternatyvių būdų skaičius;
      \end{itemize}
    \item efektyvumas:
      \begin{itemize}
        \item išmokstamumas – išmokimo laikas, perėjimas į pilno naudojimo
          etapą;
        \item robastiškumas – užduoties įvykdymas;
      \end{itemize}
    \item malonumas.
  \end{itemize}
  Poreikius formuluoja produkto aplinka: verslo tikslai, naudotojų ypatybės,
  dalykinė sritis.

  Panaudojamumo tikslai\cite[47p.]{konspektas}:
  \begin{itemize}
    \item efektyvumas;
    \item našumas;
    \item saugumas;
    \item naudingumas;
    \item išmokstamumas;
    \item lengvas atsimenamumas.
  \end{itemize}
\end{question}

\begin{question}{%
  Naudotojui palankus projektavimas.
  }{exam:question:14}
  <++>
\end{question}

\begin{question}{%
  Hierarchinė užduočių analizė.
  }{exam:question:15}
  <++>
\end{question}

\begin{question}{%
  Maketavimo paskirtis.
  }{exam:question:16}
  <++>
\end{question}

\begin{question}{%
  Paaiškinkite Gareto naudotojo potyrių projektavimo sluoksnius.
  }{exam:question:17}
  <++>
\end{question}

\begin{question}{%
  Kaip naudotojui palankus projektavimas vykdomas stambiuose projektuose?
  Pateikite atitinkamų metodikų pavyzdžius.
  }{exam:question:18}
  <++>
\end{question}

\begin{question}{%
  Paaiškinkite Normano sąveikos etapų modelį ir indėlį sąsajos
  projektavimui.
  }{exam:question:19}
  <++>
\end{question}

\begin{question}{%
  Keturi gero projektavimo principai išplaukiantys ir Normano sąveikos
  etapo modelio.
  }{exam:question:20}
  <++>
\end{question}
